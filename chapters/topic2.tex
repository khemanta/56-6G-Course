% #📘 Chapter 2: Spectrum \& Modulation in 5G

\chapter{Spectrum and Modulation in 5G}

\section{Introduction}
In wireless communication, the \textbf{spectrum} is the range of electromagnetic frequencies over which communication signals are transmitted. Efficient use of the spectrum is critical for the high-speed and reliable operation of 5G networks.

\section{Spectrum Overview}

\subsection{Licensed vs Unlicensed Bands}
\begin{itemize}
  \item \textbf{Licensed Spectrum}: Controlled by national regulatory bodies. Examples: 3.5 GHz, 28 GHz.
  \item \textbf{Unlicensed Spectrum}: Open access bands like 2.4 GHz (used by Wi-Fi).
\end{itemize}

\subsection{5G Spectrum Ranges}
\begin{itemize}
  \item \textbf{FR1 (Sub-6 GHz)}: 410 MHz to 7125 MHz
  \item \textbf{FR2 (mmWave)}: 24.25 GHz to 52.6 GHz
\end{itemize}

\section{Modulation Techniques in 5G}

\subsection{Why Modulate?}
Modulation encodes digital information onto analog carriers for transmission. 5G uses advanced modulation to support massive data throughput.

\subsection{OFDM – Orthogonal Frequency Division Multiplexing}
\begin{itemize}
  \item Splits spectrum into many orthogonal subcarriers
  \item Robust to multipath fading
  \item Basis for both LTE and 5G
\end{itemize}

\subsection{Other Modulation Techniques}
\begin{itemize}
  \item QPSK, 16-QAM, 64-QAM, 256-QAM
  \item Tradeoff: Higher order = more bits per symbol = more susceptible to noise
\end{itemize}

\section{Appendix: A2 – OFDM Signal Simulation}
A detailed hands-on simulation of OFDM signals in both time and frequency domains is provided in Appendix A2.

\section{References}
\begin{enumerate}
  \item Dahlman, E., Parkvall, S., and Skold, J. (2020). \textit{5G NR: The Next Generation Wireless Access Technology}.
  \item 3GPP TS 38.211 v16.4.0
  \item IEEE Spectrum: Understanding Modulation and OFDM
\end{enumerate}
